\documentclass[Main.tex]{subfiles}

\begin{document}
\section{Terms and Definitions}
\subsection{Agent/Robot}
In the book, \emph{Artificial Intelligence: A Modern Approach} by Russel and Norwig\cite{Russel1995}, an agent or robot is defined as, ``Anything that can be viewed as perceiving its environment through sensors and acting upon that environment through effectors.'' While this definition is sufficient in most cases, it excludes a swarm system's propensity for \emph{communication}. Therefore, an enhanced definition for a \emph{swarm robot} could be: 
\begin{quote}
Any mechanical automation capable of sensing it's surroundings, processing sensory inputs via internal computation, actuating itself or other objects in the environment based on the inputs, and communicating information with other robots around it, either directly or via \emph{stigmergy}.
\end{quote}


\subsection{Stigmergy}
Stigmergy is a term used to describe indirect communication in robot swarms. The word was first coined by Pierre-Paul Grasse, in his 1959 paper\cite{Grasse1959}, while studying insect behavior and has become a commonly used term in the swarm robotics community. 

While most robots are capable of communicating amongst themselves explicitly via infrared, \emph{Bluetooth}\texttrademark, and other wired or wireless means, most swarm algorithms try to keep such explicit communication to a minimum. This is done to maintain scalability of the system (e.g., prevent message flooding) and keep the underlying algorithm simple.

Indirect methods of communication are thus preferred when designing controllers for robot swarms, such as changing one's color or moving in a particular pattern or even just waiting at a specific position in the environment. The process of adding information to the swarm system by affecting or altering the environment rather than explicitly communicating with other agents is referred to as a stigmergic process\cite{Balch2005}. It is used in many of the swarm algorithms discussed in this paper.


\subsection{Swarm System}
A myriad of definitions are available for systems and models that exhibit ``swarming'' phenomena---the term ``swarming'' itself being re-defined in numerous cases. An interesting discussion of the nomenclature of swarm systems is available in~\cite{Beni2005,Beni2005a}.

Since our primary purpose in modeling robot swarms is to study and understand properties of the system as a whole, the term \emph{swarm system} or \emph{multi-agent system} (MAS) is not used just as a collective noun for a group of robots but instead describes the complex relationship between agents, the environment, and the tasks they are attempting to accomplish.

Multi-agent systems can be modeled at different levels of abstraction depending on the system properties we attempt to expose. When these models are used in tandem we gain the ability to both, verify and enhance the original swarm system via optimization of system parameters. We see the deployment of this strategy in the next section but first, we define the different abstraction levels used in robot swarm modeling.


\subsection{Microscopic Level}
The micro-level of a MAS model treats the individual agent as the fundamental unit of the model\cite{Lerman2001a}. Though not a requirement for this form of agent-based modeling, we generally assume that the swarm is homogeneous, i.e. every agent is running the same robot controller within it and all hardware (sensors, actuators, processors and communication devices) between robots is identical. The microscopic level then helps describe direct agent-agent interactions as well as agent-environment interaction. 

A simple method used for micro-level modeling consists of writing down the dynamics equations (equations of motion) for an individual robot and solving them to study system-level behavior. As one can imagine, these dynamics equations can become very tedious and difficult to solve for more complex swarm systems due to the high number of agents, inelastic collisions between agents and obstacles, sensor and actuator noise, etc. Therefore a more common approach to micro-level modeling involves stochastic simulation of individual robot controllers in parallel. This micro-level modeling method has been derived from the popular \emph{Gillespie} method\cite{Gillespie1976,Gillespie1977} used to model coupled chemical reactions.


\subsection{Macroscopic Level}
While micro-level models deal with system on an individual agent level, macro-level models consider the system as a whole, single entity and are used to describe collective group behavior. Macro-models for robot swarms are often phenomenological in nature. The system's parameters are derived from observing and measuring real physical phenomena and extrapolating such properties as may be deemed useful for understanding said phenomena. Macro-level models are generally represented as a system of ordinary (for non-spatial models) or partial (for spatial models) differential equations and as such are good at describing the temporal and spatial evolution of the system. They are often also referred to as population dynamics models and/or rate equations in literature.
\end{document}