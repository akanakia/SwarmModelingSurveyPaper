\documentclass[Main.tex]{subfiles}

\begin{document}

\section{Conclusion}
This paper has attempted to brief the reader on non-spatial and spatial modeling techniques in swarm robotics, at different levels of abstraction. While non-spatial models such as Gillespie simulations and PDMs are widely used in the field today, there exists no ``One method to rule them all'' when it comes to modeling any and all real robot applications. There are always robot systems and multi-agent tasks that can be designed where the primary assumptions made in modeling them using the above mentioned methods break down completely. Whether it is due to the spatial nature of the task or other complex underlying processes caused by having to deal with non-perfect hardware, environmental factors, etc. there needs to exist a strong, mathematically provable basis upon which to decide whether or not it makes sense to use the modeling techniques discussed in this paper. Such a basis does not exist as yet which is perhaps why designing accurate models for particular tasks is still so difficult and grounded in phenomenological observations versus cleaner algorithmic approaches for tackling problems commonly seen in other fields of computer science.

That said, robotics and other cyber physical systems have unique property that other fields of computer science do not---direct interaction with the real world and the laws of physics that rule supreme here. The question then becomes, what properties of the real physical system can be safely abstracted away when designing models without sacrificing accuracy and concurrency with the real world? This question is especially important in swarm robotics as so many of its applications rely on emergent behavior within the system to accomplish complex tasks using simple algorithms and interactions between thousands of individual agents\cite{Hamann2012}, e.g., self assembly of robot swarms into regular structures like bridges and honeycombs may not be realized in models that ignore the effects of friction.

With this motivation in mind, system identification and parameter discovery/estimation are areas within swarm robotics that I wish to explore further. Closely related to this is the task of robot control parameter optimization. As we saw in section~\ref{sec:opt}, knowing the system parameters, control parameters and the relationship between them (see equation~\eqref{eq:paramrel}) allows us to fully describe a swarm robot model. Optimization of the control parameters allows us to then control the behavior of the swarm, which is the end goal of this entire endeavor. 

I am also interested in studying spatial models for robot swarms that we discussed briefly in section~\ref{sec:spatial}. Their non-reliance on the assumption of spatial uniformity of the swarm makes them immensely useful for studying tasks where the positions of agents are important, such as self-assembly, pattern formation~\cite{Hsieh2006}, shape recognition~\cite{Liu2011}, boundary coverage, swarm surveillance, etc. While system parameter estimation and control parameter optimization have not been widely explored for spatial models, I think it is an area of research that can be provide very useful inferences to the entire field of swarm robotics and perhaps result in real world applications of robot swarms.
\end{document}